\documentclass[12pt]{article}
%\usepackage[left=0.8cm, right=2cm, top=2cm,bottom=2cm]{geometry}
\usepackage{mathtools}
\usepackage{amsfonts}
\usepackage{amsmath}
\usepackage{amssymb}
%\usepackage{tabu}
%\usepackage{amsthm}
%\usepackage{luacode}

\newcommand{\R}{\mathbb R}
\newcommand{\Z}{\mathbb Z}
\newcommand{\Q}{\mathbb Q}
\newcommand{\C}{\mathbb C}
\newcommand{\N}{\mathbb N}
\newcommand{\seq}[1]{\{#1 _i\}}
\newcommand{\seqn}[1]{#1_1,#1_2,\hdots,#1_n}
\newcommand{\set}[1]{\left\{#1\right\}}
\newcommand{\goesto}{\rightarrow}
\newcommand{\abs}[1]{\left| #1 \right|}
\newcommand{\inv}{^{-1}}
\newcommand{\toinfty}[1]{\lim_{#1\goesto\infty}}
\newcommand{\norm}[1]{\int \abs{#1}}
\newcommand{\inner}[2]{\left\langle{#1},{#2}\right\rangle}
\newcommand{\oneover}[1]{\frac 1 {#1}}
\newcommand{\done}{\hfill$\square$}
\newcommand{\Hom}{\text{Hom}}

\begin{document}
Let $C$ be a high res point cloud. Here are some competing models for how to produce a low res point cloud $\seq c$ from $C$.
\begin{enumerate}
	\item Sample points at random from $C$, and then add Gaussian noise.
	\item Voxelize space with low resolution. For each voxel intersecting $C$, average the points of $C$ in the voxel to produce a point in $c_i$. Variants:
		\begin{enumerate}
			\item Instead of hard voxel boundaries, we could use a Gaussian kernel to produce these local averages.
			\item A more realistic model exchanges voxels for visual cones radiating from the sensor.
		\end{enumerate}
\end{enumerate}
The first model leads to the current EM algorithm. If any blurring is applied to $C$ before the sampling process, this algorithm will just reconstruct the blurred cloud. Concretely, the reconstructed cloud is always inside the convex hull of the low res clouds.

For the second model, we construct a matrix $A$ which assigns points of $C$ to voxels (perhaps softly). First, some notation: for a point cloud $C$, let $\vec{C} \in \R^{3\abs{C}}$ be the list of coordinates of its points. Then $A\vec{C}=concat(\set{\vec{c_i}})$. The reconstruction algorithm alternates between two steps:
\begin{enumerate}
	\item Compute $A$.
	\item Set $\vec{C}=A\inv concat(\set{\vec{c_i}})$
\end{enumerate}
Now let's compare with the EM algorithm. The EM algorithm forms a ``responsibility matrix'' with the same dimensions of $A$, which assigns high res points to low res points for which they are responsible. However, the update rule looks like $\vec{C}=normalize\_rows(A^T)\cdot concat(\set{\vec{c_i}})$.
\end{document}
